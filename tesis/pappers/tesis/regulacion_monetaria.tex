\documentclass[12pt]{article}
\usepackage{amsmath}
\usepackage{tikz}
\usetikzlibrary{positioning}
\usepackage{graphicx}
\usepackage{times}         % Times New Roman
\usepackage{setspace}      % Para establecer el espaciado
\usepackage[colorlinks=true,citecolor=blue]{hyperref}
\usepackage{biblatex}
\usepackage{titlesec}         % Para modificar la apariencia de los títulos
\addbibresource{bibliografia.bib}
\setstretch{1.0}           % Espaciado simpl

\title{
    \vspace{2in} % Ajusta el espacio antes del título
    \textbf{Regulación Monetaria a través de Smart Contracts en Contextos de Desconfianza} \\[1ex]
    \large Un Enfoque Innovador para la Economía Argentina \\[2ex]
}
\author{José Carlos Carbone}
\date{\today}

\begin{document}


\maketitle
\newpage % Salto de página

\tableofcontents % Genera el índice automáticamente
\newpage % Salto de página

\begin{abstract}
    Desde la publicación del paper "Bitcoin: A Peer-to-Peer Electronic Cash System" (Nakamoto, 2008), 
    la idea de una moneda digital ha experimentado una expansión exponencial, 
    desencadenando la creación de un vasto ecosistema que continúa en expansión. 
    Este avance coincidió con un contexto económico turbulento, marcado por la crisis subprime de 2008 en Estados Unidos. 
    En este documento, se examina cómo los smart contracts pueden ser utilizados para implementar reglas monetarias 
    en contextos de alta desconfianza entre las instituciones de un estado. Se analiza la teoría subyacente de los 
    smart contracts, su aplicabilidad en la formulación de políticas monetarias, y se presenta un caso práctico de 
    implementación.
\end{abstract}

\section{Introducción} % esto es también el planteamiento del problema
% Pueden venir objetivos y objetivos específicos

\section{Marco teórico}
\section{Estado del arte}
\section{Smart Contracts y confianza}
\section{Implementación de un smart contract para regulación monetaria}
\section{Conclusiones}
\section{Prospectiva}
\end{document}